This section details the approach within ERIGrid to realise holistic testing. 
Fig.~\ref{fig:holistic_testing_revised} illustrates the main steps.

\subsection{Holistic Testing Steps}
The starting point of the envisioned procedure is at the specification of a \emph{holistic test case} (i.e., Step~1). This is derived from a scenario, a corresponding system configuration and the use cases within this setup. Consequently,
the test case is positione to identify specific test criteria, relating to the test system configuration, relevant use cases and a specific test objective. 
In an independent step, the RI involved are profiled with regards to their testing capabilities (i.e., Step~2).
As mentioned above, the procedure assumes that for such a holistic test it is not feasible to define and conduct a combined large-scale test incorporating all relevant domains and systems in one single setup. Therefore, the holistic test must be divided into sub-tests.
The sub-tests concentrate on certain components or sub-systems in total reflecting the structure of the holistic test in such a way that the sub-test results may be assembled to offer quantitative feedback on the holistic test criteria. This decomposition is performed in the first part of the mapping step (i.e., Step~3), where the interfaces and dependencies between the sub-test cases as well as the resulting requirements must be specified as well. 

In the second part of the mapping step, the descriptions of the sub-test cases, considering the RI profiles from Step~2, are used to identify the appropriate RIs capable of conducting the test for each sub-test case. 
Once the RI and tests are known the experiments can be specified, e.g., the concrete setup and design (i.e., Step~4).
Within the context of carrying out the sub-tests (i.e., Step~5) it is necessary to analyse and to exchange data and results (i.e., Step~6) between the sub-tests, based on which cross-dependencies have been identified in Step~3.
The results of all tests are analysed and combined to obtain the criteria with which the holistic test is evaluated (i.e., Step~7). Possible methods for combining results might be up-scaling or aggregating results.
Thus, the mapping between the test has two purposes: \emph{(i)} the re-use of results as an input to generate successive results, and \emph{(ii)} the combination of results from different sub-tests to obtain results of the holistic test.
To this end, dependencies between tests have to be considered beforehand.
The mapping step as well as the step of combining results of the sub-test might be an iterative approach. Before setting up and conducting the experiments the process from holistic test to RI and back should be specified as precisely as possible to minimise the effort and costs.

\begin{figure}[!t]
\centering
\includegraphics[width=0.95\columnwidth]{figures/holistic_testing_procedure_revised6.png}%\vspace{-8pt}
\caption{Descriptive elements in a holistic test specification.
Abbreviations in Step~1 are explained in Section~\ref{sec:holistic_test_definition}.}
\label{fig:holistic_testing_revised}
\end{figure}


\subsection{Holistic Test Case Specification}
\label{sec:holistic_test_definition}
A test specification aims to clarify the object under investigation, test objective, and by what means a test is to be carried out (i.e., test setup and test design): \emph{(i)} what needs to be tested, \emph{(ii)} why, and \emph{(iii)} how.  %A specific procedure aimed to test one (or more) component/functionality when the use case functions are operative.
As outlined above, the holistic testing procedure envisions a separation of the first two pillars of a test specification (i.e., test object and test objective) from the third (the means of testing). We refer to a holistic \emph{test case} as specification of the what and why of a test, without including specific limitations on test setup and test design\footnote{This kind of definition corresponds to the ICT test case definition \cite{baker2007}}.
In contrast to conventional power systems testing, this requires a more formal approach, as the intention of a test case must be unambiguously identifiable, enabling specification of a test design, and test setup in a separate step (see Section~\ref{sec:common_meta-description}).  

Another aspect of the holistic testing approach is the merger of different cultures of testing, which can be portrayed as a device-oriented culture of physical testing and a culture of testing ICT objects such as implementations of protocols and algorithms.
Rigorous formal specification of test cases as well as automated execution of tests are common in the ICT domain \cite{gnesi2012}. %, asin particular for specification and testing of protocols \kh{cite TTCN-3 related}. 
In the testing of physical components, the test object is delimited by its physical boundaries, requiring little further formalization of the test object. However, a good test specification requires insight on physical and engineering principles. Test specifications therefore tend to be domain specific and less formal. Further, much of the test design is decided by the available test setup. 
%
A challenge is therefore to formalize the complete cyber-physical system context and test criteria, to formulate a test case combining several ICT and physical components and sub-systems as well as test criteria spanning different domains.  

As illustrated in Fig.~\ref{fig:holistic_testing_revised}, we envision the specification of a \emph{holistic test case} as composed of the following description items: 
Given a smart grid scenario composed of a \emph{systems configuration} (SC) and related \emph{use cases}, as well as the intention of a \emph{test objective}, the test case intention is summarized in a  \textbf{Narrative}.
With reference to the SC, the \emph{System under Test} (\textbf{SuT}) identifies the system boundaries of an abstract test setup entailing all relevant interactions requiring investigation, and the \emph{Object under Investigation} (\textbf{OuI}) identifies to the system, subsystem or component%\kh{, application, device or function} 
 with respect to which the \emph{test criteria} will be formalized. 
The \emph{Domain(s) under Investigation} (\textbf{DoI}) identify the relevant physical or cyber-domains of test parameters and connectivity. 
With reference to use cases, the full set of \emph{Function(s) under Test} (\textbf{FuT}), and the specific \emph{Function under Investigation}  (\textbf{FuI}) are identified. 
The \emph{Purpose of Investigation} (\textbf{PoI}) formulates the test objective, also stating whether it relates to \emph{characterization, validation or verification} objectives. 
Together the above items inform the 
\textbf{Test Criteria}, which formalize the test metrics into \emph{target criteria}, \emph{variability attributes}, and \emph{quality attributes} (thresholds). 
An example of such a test case specification is provided in Section~\ref{sec:example}.

\subsection{Towards a Common Meta-Description for Sub-Tests}\label{sec:common_meta-description}

In order to support the mapping process, a common meta-description of tests is needed. This facilitates the three parts of the mapping introduced above: 
\emph{(i)} the mapping of the holistic test to sub-tests,
\emph{(ii)} the mapping of sub-tests to RI, and
\emph{(iii)} the mapping between tests reflecting the interdependencies between them. The latter refers to (re-)using results from one test in another test.
To this end, a methodology is proposed to classify tests with regard to different categories. % (see Section~\ref{sec:example}). 
For each category a classification is identified from existing testing procedures that summarizes or aggregates tests with common properties.
These categories cover different information needed for specifying sub-tests including information given in the holistic description (e.g., PoI, OuI) but also detailed information such as test setup and test design.

There are relations between the categories defining a hierarchy or order between them.
For instance, the purpose of investigation of a test determines the test criteria to be investigated. Once relations between categories have been identified dependencies between particular classes of different categories can be analysed, %e.g. which information from one cluster is needed for the related cluster.
i.e., which class from one category can be combined with which class from another category.
Firstly, this information can be used for specifying requirements within a test, i.e., information needed for the experiment specification of e.g., the design, objects, and (sub-)system to be considered within a test.
Secondly, the information can be used between tests for realising a holistic test as a combination of sub-tests. From the holistic test, requirements on sub-tests have to be specified in terms of the given categories as well as requirements on information exchange between sub-tests.
Given this categorisation and requirements the sub-tests can be derived and specified.
%Relevant information are possible relations to classes of another test, requirements on information exchange between clusters.
Thirdly, if the RI profiling is mapped to the same categories this facilitates the choice for RI capable for certain tests and thus the mapping to RI.

Perspectively, a common, formal meta-description and derived rules about relations, e.g. in form of an ontology, can support the mapping steps mentioned above. With that, conclusions can be drawn on a given test specification, e.g., if information is missing within the specification.
Furthermore, an automated decomposition of a holistic test into individual test and RI might be enabled.