In this section, first results of the work in progress are presented that aim towards the realisation of the introduced validation and testing approach.

\subsection{Assessment of Current Practice}
Information about tests that are currently conducted at RI of all ERIGrid partners have been collected with the help of a questionnaire in order to derive a first classification of tests as proposed in Section~\ref{sec:common_meta-description}.
The information have been clustered with regard to the following different categories:
\emph{(i)} purpose of investigation, \emph{(ii)} test setup, \emph{(iii)} test criteria, \emph{(iv)} test design, and \emph{(v)} object of investigation.
The work has been separated into five working groups each responsible for one of the categories. The aim has been to identify different types of tests from the questionnaires as clusters and give a description of the common properties.
After this initial activity, another working group has been initialised for aligning the results of the five working groups regarding definitions and relations between clusters. 
The result has been a first consolidated description for classes of the categories.

It has been identified that additional information and categorisation is necessary on interfaces or connections of one test to other components or (sub-)systems. This will facilitate the mapping between different sub-tests.
In a next step, the tests covered in the questionnaires will be sorted to the classes in order to obtain a first impression about relations between classes of different categories.
Giving a holistic test specification, the classification can be adapted and refined and relations between classes and categories be detailed for classes that are relevant to this specific holistic test.

\subsection{Description of a Cyber-Physical System Configuration}\label{subsec:testcasedescription}
A cyber-physical system configuration in a smart grid context comprises physical components and devices, as well as various forms of ICT objects and relevant abstract components (e.g., markets, services). A continuum between concrete and abstract objects and their interconnections needs to be formally represented to specify a holistic (multi-domain) system under test. We define systems, domains, components, etc. as illustrated in Fig.~\ref{fig:examplediagram}; definitions wrt. standards / lexical terms.
%\kh{This Section can be extended with: the definitions and references to them can be included; a discussion of its analogy to SYSML component diagrams (but simpler).}

\subsection{Related Work on System Testing}
The testing objectives in a holistic testing procedure should be viewed in context of a systems design procedure, as the goals and conditions of a test vary at different stages of development. For example during an early stage of systems development, a test may aim at the characterization and algorithm's performance, whereas at a later, more mature, stage also conformance with specific standards may be required. The view of system maturity and corresponding testing needs are outlined in different development models (integrated specification development and testing), such as test-driven, or agile development in ICT domain, or the well-known V-model or the W-model \cite{beizer1990}. %\kh{could be extended with specific reference to literature  and Standards, ...}